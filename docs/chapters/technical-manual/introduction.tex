\subsection{¿Qué es aAUTOMATOR?}
\emph{aAUTOMATOR} es una herramienta para el desarrollo fácil y rápido
de agentes software a medida destinados a la extracción de información
de la Web \cite{aAUTOMATOR}. Estas aplicaciones, denominadas robots,
recorren y analizan las páginas web extrayendo y combinando la
información existente según el formato especificado por el
usuario. aAUTOMATOR se compone de una herramienta visual para el
diseño y ejecución de robots que evita la necesidad de disponer de
conocimientos avanzados de lenguajes de programación.

En la actualidad la \emph{WWW} presenta una cantidad ingente de
información. Concretamente, en el campo de la bioinformática y la
biología computacional cabe destacar la amplia disponibilidad de
recursos en-línea.  Un problema al que nos enfrentamos es que esta
información es difícil de procesar y extraer automáticamente, ya que
la inmensa mayoría de esta información está destinada a ser leída por
humanos. Mientras existen esfuerzos, como la web semántica, para
codificar toda esa información de una manera más formal y tratable por
computadoras; hoy por hoy no es posible acceder a la mayor parte de la
información de la web por medio de procedimientos automáticos.

En este sentido, la Web es el medio que posibilita el acceso al nuevo
conocimiento generado por diferentes grupos de investigación de todo
el mundo y que, sin embargo, continúa presentando importantes retos
relacionados con el acceso y la extracción de información
útil. Entre otros, cabe mencionar los siguientes inconvenientes:
\begin{itemize}
\item Elevada cantidad de información. Las interfaces web de acceso a
  información genómica suelen generar como resultado datos de elevado
  nivel de detalle que, si bien en muchos casos es lo buscado, en
  otros únicamente resulta de interés una parte reducida del
  resultado.
\item Múltiples formatos de presentación. El aspecto y estructura de
  los resultados es diferente en función de la fuente de información a
  la que se accede.
\item Información distribuida en diferentes lugares. Suele ser muy
  habitual que la información buscada no se encuentre únicamente en un
  lugar, sino que sea necesario el acceso a múltiples fuentes de
  información realizando búsquedas y copiando/pegando resultados que
  se dirigirán de forma manual hacia nuevas búsquedas en otros
  lugares.
\end{itemize}

Ante este panorama una de las mayores iniciativas es la introducción
de la web semántica: RDF, OWL, SPARQL, Servicios Web SOAP/WSDL,
etc. No obstante, el uso de estas herramientas dista todavía de
ser algo generalizado debido sobre todo a la antigüedad de las webs o
a que no ha sido considerado su acceso por parte programas software.

Nos vemos abocados al desarrollo de aplicaciones software que acceden
a las páginas web seleccionando, formateando y combinando la
información extraída con el fin de generar una versión personalizada
que cumpla los requisitos. la única alternativa para la creación de
estas aplicaciones implica un esfuerzo a mayores durante la fase de
extracción de información, que se basa en el análisis del código HTML
en busca de patrones de cadenas de texto. A mayores, se hacen
necesarios conocimientos de programación para la construcción de este
tipo de `agentes', tanto si se utilizan tecnologías de la web
semántica, como si se analiza texto HTML.

En este contexto surge aAUTOMATOR, una herramienta altamente flexible
que permite la creación de \emph{robots}\footnote{No son más que
  agentes de software, por brevedad en aAUTOMATOR se llaman robots.}
para la extracción de información en sitios web.

aAUTOMATOR se compone de dos partes fundamentales:
\begin{description}
\item[Herramienta de edición robots.] Permite el diseño y la ejecución
  de robots en un entorno visual y amigable. El objetivo de esta
  herramienta es la creación de robots sin conocimientos de
  programación, una de las características clave

\item[aAUTOMATOR API.] Librería sobre la que se basa el componente
  anterior. Permite la creación, instanciación y ejecución de robots a
  nivel de programación desde otras aplicaciones.
\end{description}

El interés de \emph{DARE} está en este último componente. \emph{DARE} está
enfocado para ser usado directamente por programadores ofreciendo todo
el poder de \emph{aAUTOMATOR API} en una forma conveniente.

Para entender el funcionamiento de DARE y su terminología vamos a
necesitar adentrarnos en el funcionamiento de \emph{aAUTOMATOR}.

\subsubsection{Estructura de un robot}

Un robot está compuesto por elementos denominados
\emph{transformadores}. Cada uno de ellos realizan una función simple y bien
definida. Un transformador recibe un vector de cadenas de texto,
aplica una serie de transformaciones y devuelve otro vector de cadenas
de texto como salidas. Ejemplos de transformadores incluidos en
aAUTOMATOR son la conversión de URLs a su contenido HTML, búsquedas de
texto, reemplazos, etc.

Un robot se puede contemplar como un grafo acíclico dirigido,
DAG\cite{DAG}, en el que los nodos son los transformadores y las
aristas indican que sus resultados se dirigen a otro transformador.

Sin embargo, la representación real es en forma de árbol. El
transformador realiza su operación sobre el vector de entrada,
proporciona su vector de salida a cada hijo y combina el resultado de
éstos para obtener el resultado final del transformador.

%TODO: introducir figura con vista en arbol y vista logica

Un transformador se define por el comportamiento que realiza sobre el
vector de entrada. El modo en el que se proporciona el vector de
salida a cada hijo y como se recombina el resultado de éstos es
parametrizable.

\begin{description}
  \item{BRANCH TYPE.} Indica como el transformador proporciona su
    vector de salida a sus hijos. Nos ofrece tres opciones:
    \begin{itemize}
      \item CASCADE. El vector de salida de proporciona únicamente al
        primer hijo, el resultado del primer hijo al segundo y así
        sucesivamente. El resultado generado por el último hijo será
        el resultado final del transformador.
      \item BRANCH-DUPLICATED. El vector de salida se proporciona a
        todos los hijos, cada hijo recibe una copia.
      \item BRANCH-SCATTERED. El vector de salida se reparte entre los
        hijos, correspondiendo la primera posición al primer hijo, la
        segunda posición al segundo hijo y así sucesivamente. Si el
        número de elementos del vector de salida es mayor que el
        número de hijos, se vuelve a repetir el proceso con la parte
        todavía no repartida empezando por el primer hijo.
    \end{itemize}
  \item{MERGE MODE.} Sólo tiene validez en el caso de que
    BRANCH-TYPE no sea CASCADE. Indica como se recombina el resultado
    de los hijos para formar el resultado final. Nos ofrece tres
    opciones:
    \begin{itemize}
      \item ORDERED. El resultado final será un único vector resultado
        de concatenar el resultado final de todos los hijos.
      \item SCATTERED. El resultado final será un único vector
        resultado de ir tomando un elemento del resultado final de
        cada hijo hasta consumir los resultados de todos los hijos. Es
        decir, funciona igual que ORDERED, salvo que se toman las
        cadenas de los hijos de forma alternada. El resultado final
        sería un vector en el que en primer lugar irían las primeras
        posiciones de los hijos, luego las segundas posiciones, etc.
      \item COLLAPSED. El resultado sería un vector con una única
        posición en el que se concatenan todas y cada una de las
        cadenas de los resultados finales de los hijos.
    \end{itemize}
\end{description}

Ya podemos observar como a partir de transformadores con
comportamientos sencillos, por medio de composición y parametrización
se pueden alcanzar comportamientos complejos. Sin embargo, para
alcanzar los objetivos deseados, necesitamos permitir la
implementación de bucles.

Cualquier transformador se puede ejecutar de manera cíclica si se le
indica con el parámetro LOOP a \emph{true}. Un transformador en modo
LOOP se comporta del siguiente modo:

\begin{itemize}
  \item El primer hijo del transformador se utiliza para controlar el
    bucle. Sus resultados no serán incluidos en el resultado final. A
    este primer hijo se le proporciona el vector de salida generado
    por el transformador (no el resultado final).
  \item El transformador examina el vector de salida de este primer
    hijo. Si no proporciona ningún resultado, el bucle ha de
    terminar. En caso contrario la salida del primer hijo se empleará
    como nuevo vector de entrada.
  \item Una vez finalizado el bucle, se produce el resultado
    final. Por cada iteración del bucle se produce un resultado
    intermedio empleando el MERGE-MODE especificado. Una vez
    finalizado el bucle se toma el resultado de cada iteración y se
    añade al resultado final. De modo que las primeras posiciones se
    corresponderán al resultado de la primera iteración, las
    siguientes a la segunda iteración, etc. Por ejemplo, si el
    MERGE-MODE es COLLAPSED el resultado final tendrá una posición por
    cada iteración.
\end{itemize}

Los bucles son imprescindibles a la hora de procesar resultados
devueltos en varias páginas.

\subsubsection{Transformadores incluidos}

aAUTOMATOR trae incorporados una serie de transformers predefinidos.
\begin{table}[!hbp]
\small{
\begin{tabular}{l p{6cm} p{5cm}}
Clase Java & Descripción & Parámetros \\ \hline

SimpleTransformer & La entrada es igual a la salida & \\ \hline
URLRetriever & Cada cadena del vector de entrada se interpreta como
una URL y se obtiene un vector con las líneas de texto de las que está
compuesto el recurso & \\ \hline
URLDownloader & Cada cadena del vector de entrada se interpreta como
una URL y su contenido se guarda en un fichero & \emph{DownloadPath:}
directorio donde guardar los ficheros \linebreak[4] \emph{overwrite:}
sobreescribir ficheros ya existentes \\ \hline
\end{tabular}
}
\caption{Transformers incluidos}
\end{table}
