 \section{Instalación}
Veamos primero como instalar DARE en un único nodo. Para ello
utilizaremos la \emph{gema} foreman. Los siguientes pasos de han
comprobado en un Ubuntu 11.10. Las instrucciones deberían ser
similares para cualquier sistema Linux actual, aunque podrían diferir
en detalles. Como resultado de seguir estos pasos se debería tener
DARE ejecutándose en un único nodo con dos procesos \emph{web} y dos
procesos \emph{worker}.

\subsection{Dependencias}
\begin{description}
\item[MongoDB.] Para instalar MongoDB en Ubuntu recomendamos seguir
  las instrucciones presentes en
  \url{http://www.mongodb.org/display/DOCS/Ubuntu+and+Debian+packages}. Una
  vez instalado \verb+mongo --version+ debería producir algo similar
  a: \verb+MongoDB shell version: 2.0.3+.
\item[NginX.] Vamos a utilizar el servidor web NginX para actuar como
  un proxy inverso con balanceo de carga. Para instalar efectuar:
  \verb+sudo apt-get install nginx+.
\item[Ruby Gems.] Sirve para instalar
  \emph{gems}\cite{RUBY_GEMS}\footnote{Formato de distribución de
    librerías Ruby}. Se necesita para instalar Foreman.  Se pude
  instalar utilizando el comando:
  \verb+sudo apt-get install rubygems1.8+ Otra opcion es utilizar
  \emph{rvm}, para ello seguir las instrucciones en
  \url{http://beginrescueend.com/}.
\item[Foreman.] Es una gem que instala en el sistema el comando
  Ruby. Para instalar efectuar: \verb+gem install foreman+.
\end{description}

\subsection{Pasos}

\subsubsection{Configurar DARE}
\begin{enumerate}
  \item Crear usuario dare con el que se ejecutará el servicio. 
    \begin{verbatim}
sudo addgroup --system dare
sudo adduser --system --no-create-home --ingroup dare -shell '/bin/bash' dare
    \end{verbatim}
  \item Crear directorio donde se encontrarán los ejecutables.
    \begin{verbatim}
sudo mkdir /opt/dare
    \end{verbatim}
  \item Copiar ejecutables DARE. \verb+target-location+ es el
    directorio target dentro de tu distribución DARE. Allí deberían
    estar los jars y el war de la aplicación.
    \begin{verbatim}
sudo mkdir /opt/dare
sudo cp <target-location>/* /opt/dare
sudo cp <target-location>/.foreman /opt/dare
sudo chown -R dare:dare /opt/dare
    \end{verbatim}

  \item Exportar la aplicación para poder ser ejecuta desde el sistema
    de manejo de procesos del sistema. En el caso de Ubuntu se emplea
    Upstart\bibitem{UPSTART}. Si se está utilizando rvm hay que hacer
    rvmsudo en vez de sudo.

    \verb+sudo foreman export upstart /etc/init -f Procfile.production -u dare+

    En el fichero \verb+.foreman+ se encuentran algunos valores por
    defecto. Como el nombre de la aplicación, el puerto y la
    concurrencia. Si se desea cambiar algún parámetro se puede cambiar
    en este fichero o pasarlo como parámetro a foreman. Por ejemplo
    para exportar un script upstart que lance tres procesos worker y
    uno process web habría que invocar foreman del siguiente modo:

    \begin{verbatim}
sudo foreman export upstart /etc/init -f Procfile.production -c 'web=1,worker=3'
    \end{verbatim}
    Ejecutar \verb+foreman help export+ para ver más opciones.
\end{enumerate}

Ahora podemos hacer: 
\begin{verbatim}
# para lanzar dare:
sudo dare start
# para parar dare
sudo dare stop
# para parar sólo el worker-2
sudo dare stop worker-2
# para lanzar el worker-2
sudo dare start worker-2
\end{verbatim}
Una vez lanzado dare debería haber dos procesos web y dos procesos
worker escuchando en los puertos correspondientes. Los procesos web
escucharán en los puertos 5000 y 5001. Obviamente los usuarios de DARE
esperan que la aplicación esté escuchando en un dominio con el puerto
estándar. Para ello debemos de configurar un proxy inverso que balancee
la carga entre los procesos web configurados. 

Si deseamos instalar DARE en más nodos repetimos este proceso en cada
nodo. Sólo habría que instalar Nginx en uno de los nodos y hay que
acordarse de listar todos los procesos en web en \verb+upstream dare+.

\subsubsection{Configurar Nginx}

Vamos a utilizar Nginx como balanceador de carga. En el directorio
\verb+/etc/nginx/+ se encuentra su configuración.

Primero creamos una nueva configuración, por ejemplo creando un
fichero dare en \verb+/etc/sites-available/default+ con este contenido:
\begin{verbatim}
upstream dare {
# list of launched web processes
   server 127.0.0.1:5000;
   server 127.0.0.1:5001;
}

server {
       listen 80 default_server;
       server_name       "";

       location / {
                proxy_pass http://dare;
                include proxy_params;
       }
}
\end{verbatim}

A continuación creamos un enlace a esta configuración:

\verb+sudo ln -s /etc/nginx/sites-available/dare /etc/nginx/sites-enabled/dare+

Por último hay que eliminar el enlace a la configuración por defecto:

\verb+sudo rm /etc/nginx/sites-enabled/default+.

\section{Especificación API REST}

A continuación describimos la API REST de manera detallada. Esta
documentación debería permitir crear librerías para DARE en otras
plataformas.

\subsection{Puntos entrada}
El primer dato que necesitamos es conocer los puntos de entrada a la
aplicación. Para ello primero debemos conocer la URL en la que está
instalada la aplicación, \verb+<base-url>+ a partir de ahora.

A partir de este dato se puede acceder a varios puntos de entrada.

\begin{table}[hp]
\begin{tabularx}{\textwidth}{l X}
\multirow{2}{*}{Acción}
 & POST \verb+<base-url>/robot/execute+ \\
 & Crea el robot proporcionado y lanza una solicitud de
ejecución.\\ \hline
\multirow{2}{*}{Petición}
 & application/x-www-form-urlencoded \\
 & Un campo \emph{robot} con una cadena de
  minilenguaje válida. \newline
  Cero, uno o varios campos \emph{input} que formarán el vector de
  entrada.\\ \hline
\multirow{2}{*}{Respuesta}
& 201 Created \\
& Tiene un campo location con la url en donde se pondrá la \emph{execution result}
una vez finalizada.
Esta \emph{execution result} no tiene porque existir todavía.\\
\end{tabularx}
\caption{Ejecutar nuevo Robot}
\label{execute_new_robot}
\end{table}

\begin{table}[hp]
\begin{tabularx}{\textwidth}{l X}
\multirow{2}{*}{Acción}
 & POST \verb+<base-url>/robot/create+ \\
 & Crea el robot proporcionado.\\ \hline
\multirow{4}{*}{Petición}
 & application/x-www-form-urlencoded \\
 & Un campo \emph{minilanguage} con una cadena de
  minilenguaje válida.\\ \cline{2-2}
 & application/xml \\
 & Un robot en el formato XML de \emph{aAUTOMATOR}.\\ \hline
\multirow{2}{*}{Respuesta}
& 201 Created \\
& Tiene un campo location con la url del robot creado.
Esta \emph{execution result} no tiene porque existir todavía.\\
\end{tabularx}
\caption{Crear nuevo Robot}
\label{create_new_robot}
\end{table}
\newpage

\subsection{Recurso Robot}

Por medio del recurso Crear nuevo Robot --- cuadro
\ref{create_new_robot}, pág.~\pageref{create_new_robot}--- se obtiene
la URL a un nuevo Robot. Esta URL actualmente es de la forma
\verb+<base-url>/robot/{robot-code}+, pero esto está sujeto a
cambio. La aplicación cliente debería trabajar con esta URL al crear
un nuevo Robot. A esta URL la llamaremos \verb+<robot-url>+. A partir
de esta URL se pueden realizar varias acciones.

\begin{table}
\begin{tabularx}{\textwidth}{l X}
\multirow{2}{*}{Acción}
 & GET \verb+<robot-url>+ \\
 & Muestra la información del robot indicado en alguno de los formatos
soportados. El formato deseado se indica mediante la cabecera Accept.\\ \hline
\multirow{4}{*}{Respuesta}
& application/json \\
& Robot en formato JSON. Ver cuadro~\ref{robot_json_representation}, pág.~\pageref{robot_json_representation}. \\ \cline{2-2}
& application/xml \\
& Robot en formato XML. Ver cuadro~\ref{robot_xml_representation}, pág.~\pageref{robot_xml_representation}. \\
\end{tabularx}
\caption{GET robot}
\label{get_robot}
\end{table}

\begin{table}
\begin{tabularx}{\textwidth}{l X}
\multirow{2}{*}{Acción}
 & DELETE \verb+<robot-url>+ \\
 & Elimina el Robot más las ejecuciones asociadas, tanto normales como
periódicas.\\ \hline
\multirow{1}{*}{Respuesta}
& 200 OK \\
\end{tabularx}
\caption{DELETE robot}
\label{delete_robot}
\end{table}

\begin{table}
\begin{tabularx}{\textwidth}{l X}
\multirow{2}{*}{Acción}
 & POST \verb+<robot-url>/execute+ \\
 & Lanza una solicitud de ejecución para el robot indicado.\\ \hline
\multirow{2}{*}{Petición}
 & application/x-www-form-urlencoded \\
 & Cero, uno o varios campos \emph{input} que formarán el vector de entrada.\\ \hline
\multirow{2}{*}{Respuesta}
& 201 Created \\
& Campo location con la url a la execution result creada.\\
\end{tabularx}
\caption{Execute robot}
\label{execute_robot}
\end{table}

\begin{table}
\begin{tabularx}{\textwidth}{l X}
\multirow{2}{*}{Acción}
 & POST \verb+<robot-url>/periodical+ \\
 & Crea una ejecución periódica. Una ejecución periódica es como una
ejecución normal pero que se ejecuta cada cierto periodo de tiempo.\\ \hline
\multirow{2}{*}{Petición}
 & application/x-www-form-urlencoded \\
 & Campo \emph{period} con el formato <numero><tipo> donde tipo puede
ser d, h o m para especificar días, horas o minutos respectivamente.\newline
 Cero, uno o varios campos \emph{input} que formarán el vector de entrada.\\ \hline
\multirow{2}{*}{Respuesta}
& 201 Created \\
& Campo location con la url a la execution result creada.\\ \hline
\end{tabularx}
\caption{Create periodical}
\label{create_periodical}
\end{table}
\newpage

\subsection{Recurso Resultado Ejecución}

Utilizando los recursos para realizar ejecuciones
---cuadro~\ref{execute_new_robot}, pág.~\pageref{execute_new_robot} y
cuadro~\ref{execute_robot}, pág.~\pageref{execute_robot}--- se obtiene
la URL a un nuevo Resultado de Ejecución. Esta URL es actualmente de
la forma \verb+\<base-url>/result/{code}+, pero esto está sujeto a
cambio. La aplicación cliente debería trabajar con la URL obtenida al
crearse una nueva ejecución. A esta URL la llamaremos
\verb+<execution-url>+.

Las ejecuciones de Robots llevan cierto tiempo realizarse. Por ello,
mientras la ejecución todavía no se haya completado se devuelve un
resultado \verb+204 NO CONTENT+. Es responsabilidad de la aplicación
cliente seguir intentándolo hasta que el resultado de la ejecución
esté disponible.

\begin{table}
\begin{tabularx}{\textwidth}{l X}
\multirow{2}{*}{Acción}
 & GET \verb+<execution-url>+ \\
 & Muestra la información de la ejecución indicada en alguno de los
formatos soportados. Indicar el formato deseado mediante la cabecera Accept.\\ \hline
\multirow{6}{*}{Respuesta}
& 204 No Content \\
& Si la ejecución todavía no ha sido completada.\\ \cline{2-2}
& application/json \\
& Ejecución en formato JSON. Ver cuadro~\ref{execution_json_representation}, pág.~\pageref{execution_json_representation}.\\ \cline{2-2}
& application/xml \\
& Ejecución en formato XML. Ver cuadro~\ref{execution_xml_representation}, pág.~\pageref{execution_xml_representation}. \\
\end{tabularx}
\caption{GET Execution Result}
\label{get_execution_result}
\end{table}

\begin{table}
\begin{tabularx}{\textwidth}{l X}
\multirow{2}{*}{Acción}
 & DELETE \verb+<execution-url>+\\
 & Elimina la Ejecución indicada.\\ \hline
\multirow{1}{*}{Respuesta}
 & 200 OK \\
\end{tabularx}
\caption{DELETE Execution Result}
\label{delete_execution_result}
\end{table}
\newpage

\subsection{Recurso Ejecución Periódica}

Utilizando el recurso para crear una ejecución periódica
---cuadro~\ref{create_periodical}, pág.~\pageref{create_periodical}---
se obtiene la URL que identifica una Ejecución Periódica. Esta URL es
actualmente de la forma
\verb+\<base-url>/result/{periodical-execution-code}+, pero esto está
sujeto a cambio. La aplicación cliente debería trabajar con la URL
obtenida al crearse una nueva ejecución periódica. A esta URL la
llamaremos \verb+<periodical-url>+.

\begin{table}
\begin{tabularx}{\textwidth}{l X}
\multirow{2}{*}{Acción}
 & GET \verb+<periodical-url>+ \\
 & Muestra la información de la ejecución periódica indicada en alguno de los
formatos soportados. El formato deseado se indica mediante la cabecera Accept.\\ \hline
\multirow{4}{*}{Respuesta}
& application/json \\
& Ejecución Periódica en formato JSON. Ver
cuadro~\ref{periodical_execution_json_representation}, pág.~\pageref{periodical_execution_json_representation}.\\ \cline{2-2}
& application/xml \\
& Ejecución Periódica en formato XML. Ver
cuadro~\ref{periodical_execution_xml_representation}, pág.~\pageref{periodical_execution_xml_representation}. \\
\end{tabularx}
\caption{GET Periodical Execution}
\label{get_periodical_execution}
\end{table}

\begin{table}
\begin{tabularx}{\textwidth}{l X}
\multirow{2}{*}{Acción}
 & DELETE \verb+<execution-url>+\\
& Elimina la Ejecución Periódica indicada.\\ \hline
\multirow{1}{*}{Respuesta}
 & 200 OK \\
\end{tabularx}
\caption{DELETE Ejecución Periódica}
\label{delete_periodical_execution}
\end{table}

\begin{table}
\begin{minted}{javascript}
{robotXML: '<?xml version="1.0" encoding="UTF-8" standalone="no"?>
  <robot version="1.0"> .... ... </robot>',
code: '283cceff-8e05-48ed-8aca-421facd81848',
robotInMinilanguage': 'url',
creationDateMillis: 1332352716333L}
\end{minted}
\caption{Representación JSON Robot}
\label{robot_json_representation}
\end{table}

\begin{table}
\begin{minted}{xml}
<stored-robot>
  <code>283cceff-8e05-48ed-8aca-421facd81848</code> <!--UUID code -->
  <creationDateMillis>1332352716333</creationDateMillis>
  <robot version="1.0">
   <!--... Robot in aAUTOMATOR format -->
  </robot>
<robotInMinilanguage>url</robotInMinilanguage>
</stored-robot>
\end{minted}
\caption{Representación XML Robot}
\label{robot_xml_representation}
\end{table}

\begin{table}
\begin{minted}{javascript}
{executionTime: 110,
inputs: ['http://www.google.es'],
resultLines: ['first output line', 'second output line'],
createdFrom: 'http://localhost:5000/robot/283cceff-8e05-48ed-8aca-421facd81848',
creationDateMillis: 1332353014267L}
\end{minted}
\caption{Representación JSON Resultado Ejecución}
\label{execution_json_representation}
\end{table}

\begin{table}
\begin{minted}{xml}
<result>
  <resultLines>
  <line>
    first output line
    </line>
    <line>
    second output line
    </line>
  </resultLines>
  <executionTime>110</executionTime>
  <creationDateMillis>1332353014267</creationDateMillis>
  <createdFrom>
  http://localhost:5000/robot/283cceff-8e05-48ed-8aca-421facd81848
  </createdFrom>
  <inputs>
    <input>http://www.google.es</input>
  </inputs>
</result>
\end{minted}
\caption{Representación XML Resultado Ejecución}
\label{execution_xml_representation}
\end{table}

\begin{table}
\begin{minted}{javascript}
{'code': '391ad217-63cb-4699-a571-602c431b4ffa',
'periodAmount': 1,
'periodUnit': 'days',
'inputs': ['http://www.google.es'],
'lastExecutionResult':
     {'executionTime': 435,
      'resultLines': ['first output line', 'second output line'],
      'creationDateMillis': 1332712521399L},
'robot': 'http://localhost:5000/robot/283cceff-8e05-48ed-8aca-421facd81848',
'creationDateMillis': 1332352996708L}
\end{minted}
\caption{Representación JSON Ejecución Periódica}
\label{periodical_execution_json_representation}
\end{table}

\begin{table}
\begin{minted}{xml}
<periodical-execution>
  <code>391ad217-63cb-4699-a571-602c431b4ffa</code>
  <creationDateMillis>1332352996708</creationDateMillis>
  <robot>
  http://localhost:5000/robot/283cceff-8e05-48ed-8aca-421facd81848
  </robot>
  <periodUnit>days</periodUnit>
  <periodAmount>1</periodAmount>
  <inputs>
  <input>http://www.google.es</input>
  </inputs>
  <lastExecutionResult>
    <resultLines>
      <line>
      first output line
      </line>
      <line>
      second output line
      </line>
    </resultLines>
    <executionTime>435</executionTime>
    <creationDateMillis>1332712521399</creationDateMillis>
  </lastExecutionResult>
</periodical-execution>
\end{minted}
\caption{Representación XML Ejecución Periódica}
\label{periodical_execution_xml_representation}
\end{table}

\newpage

\section{Línea comandos}
\subsection{Ejemplos}
